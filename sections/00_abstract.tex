Underwater imaging is critical for various applications such as marine biology, underwater archaeology, and pipeline inspection. However, backscatter from suspended particles significantly degrades image quality, posing a substantial challenge. This paper addresses the challenge of mitigating underwater backscatter in real-time imaging by developing a backscatter cancellation system using a Raspberry Pi Single Board Computer. The paper proposes a solution that leverages a combination of image processing techniques, including and revolving around the Canny edge detection algorithm, to accurately detect and segment backscatter particles. To evaluate the system, the paper develops a bubble backscatter simulator, and a lossless video recorder for controlled testing and real-world footage analysis. Performance tests revealed an average frame processing latency of \SI{2.6}{\milli\second}, outperforming systems in previous work that operate on more powerful hardware. Attempts to enhance performance using multiprocessing and a real-time operating system (RTOS) patch, however, resulted in increased latency due to the overhead of Inter-Process Communication (IPC) and frequent kernel context switching. These findings suggest that simpler single-core implementations may offer superior performance for I/O-bound tasks. The results demonstrate significant progress in reducing underwater backscatter, with potential applications across various underwater imaging tasks. Future work will focus on hardware improvements and further software optimisation to refine system performance.
