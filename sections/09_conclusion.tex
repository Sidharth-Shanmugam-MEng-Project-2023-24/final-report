This project set out to design, implement, and evaluate a real-time backscatter cancellation system for underwater imaging, leveraging an RPi SBC for portability and accessibility. The project began by developing the toolsets for the final system, including a backscatter simulation to generate a synthetic ground truth to test and validate the final system, and a lossless recording script to capture underwater footage to test the final system without requiring underwater deployment. These toolsets provided crucial assistance in the final system's development.

The project achieves the objective of precise backscatter particle segmentation, even in varying environmental conditions, by employing an image processing pipeline centred around the Canny edge detection algorithm, with an adjustment to replace the histogram equalisation stage with binary thresholding, and a simple minimum enclosing circle segmentation method. The project explores methodologies to reduce system latency with a real-time kernel and multiprocessing for increasing system throughput to achieve the latency targets, uncovering mixed results, including the adverse effects of the real-time kernel patch for Linux and Python multiprocessing. However, the single-core system achieves a commendable \SI{2.6}{\milli\second} average frame processing latency, a clear improvement over previous research.

Underwater testing of the system, with the lossless recording program, uncovered the drastic parallax and distortion effects due to the submersible housing construction, which resulted in the project's shift from focusing on the development of a final working system, to real-time software optimisation, to balance the short project time duration. Therefore, driving the DLP projector, including functionality to control and establish a fixed system frame rate, was no longer an actionable item. Appendix \ref{gantt} illustrates the evolution of the project schedule using Gantt charts.

In conclusion, this project successfully delivers a functional and efficient backscatter cancellation system, with notable improvements in real-time processing speed. The insights gained from the performance evaluation highlight the intricacies of optimising such systems and suggest that, for specific applications, simpler single-core implementations may offer superior performance compared to more complex multiprocessing or RTOS-enhanced solutions. Future work must first focus on completing this system's objectives for DLP-driven projections, fine-grain framerate controls, a more realistic backscatter simulation, and verifying the system's performance with the synthetic ground truth. After that, the following research can drastically improve the system: (a) FPGA implementation for accelerated image processing and DLP projector driving by harnessing the inherently multithreaded architecture, (b) improved submersible housing, mitigating the component offsets with high-precision alignment and a beamsplitter to eliminate parallax by co-location, and finally, (c), a predictive system to track and estimate the future location of backscatter particles to mitigate backscatter movement against system latencies, with options to incorporate machine-learning technologies, using a more realistic backscatter simulation for training data.
