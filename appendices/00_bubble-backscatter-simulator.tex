\subsection{Description of Class Fields \& Methods}
\label{sim_tbls}
\begin{table}[H]
    \centering
    \begin{tabularx}{\linewidth}{c | X}
        Name    &   Description\\
        \hline
        \hline
        \texttt{WINDOW\_WIDTH}          &   The maximum width of the simulation window.\\
        \hline
        \texttt{WINDOW\_HEIGHT}         & The maximum height of the simulation window.\\
        \hline
        \texttt{SIMULATION\_FPS}        & The frame rate to simulate.\\
        \hline
        \texttt{MAX\_VELOCITY\_X}       & The maximum possible x-axis velocity for particles.\\
        \hline
        \texttt{MIN\_VELOCITY\_X}   & The minimum possible x-axis velocity for particles.\\
        \hline
        \texttt{MAX\_VELOCITY\_Y}   & The maximum possible y-axis velocity for particles.\\
        \hline
        \texttt{MAX\_VELOCITY\_X}       & The minimum possible y-axis velocity for particles.\\
        \hline
        \texttt{VELOCITY\_CHANGE\_RATE}     & The rate at which the velocity changes.\\
        \hline
        \texttt{MAX\_RADIUS}    & The maximum possible spawn-time particle radius.\\
        \hline
        \texttt{MIN\_RADIUS}    & The minimum possible spawn-time particle radius.\\
        \hline
        \texttt{HEIGHT\_RADIUS\_GROW\_MULTIPLIER}   & The particle radius growth rate as the particle rises.\\
        \hline
        \texttt{BG\_COLOUR}  & The background colour of the simulation window.\\
        \hline
        \texttt{PARTICLE\_COLOUR}  & The particle colour the simulation backscatter.\\
        \hline
        \texttt{MAX\_PARTICLES}  & The maximum possible particles in each frame.\\
        \hline
        \texttt{CONSTANT\_PARTICLE\_GENERATION}  & Option to constantly spawn particles.\\
        \hline
        \texttt{PARTICLE\_RANDOMISE\_VELOCITY\_PROB}  & The probability of particle movement randomisation.\\
        \hline
    \end{tabularx}
    \caption{Simulation parameter constants and their descriptions from the Bubble Backscatter Generator program.}
    \label{table:simbubbleclassparams}
\end{table}

\begin{table}[H]
    \centering
    \begin{tabularx}{\linewidth}{c | X}
        Name    &   Description\\
        \hline
        \hline
        \texttt{id}          &   The unique ID of the particle, generated by \texttt{uuid.uuid4()}.\\
        \hline
        \texttt{x}                       &   The x-axis coordinate position of the particle.\\
        \hline
        \texttt{y}                       &   The y-axis coordinate position of the particle.\\
        \hline
        \texttt{radius}                  &   The radius of the particle.\\
        \hline
        \texttt{velocity\_x}             &   The current x-axis velocity of the particle.\\
        \hline
        \texttt{velocity\_y}             &   The current y-axis velocity of the particle.\\
        \hline
        \texttt{target\_velocity\_x}     &   The target future x-axis velocity of the particle.\\
        \hline
        \texttt{target\_velocity\_y}     &   The target future y-axis velocity of the particle.\\
        \hline
    \end{tabularx}
    \caption{Fields and their descriptions from the Bubble class of the Bubble Backscatter Generator program.}
    \label{table:simbubbleclassfields}
\end{table}

\begin{table}[H]
    \centering
    \begin{tabularx}{\linewidth}{c | X}
        Name    &   Description\\
        \hline
        \hline
        \texttt{\_\_init\_\_()}     &   Initialises a new backscatter \texttt{Bubble} object, generating a unique \texttt{id} with \texttt{uuid.uuid4()}, a random \texttt{x} starting position, \texttt{y} starting position at the bottom of the screen, random \texttt{target\_velocity\_x} and \texttt{target\_velocity\_y} velocities which are set equal to the initial velocities, \texttt{velocity\_x} and \texttt{velocity\_y}.\\
        \hline
        \texttt{randomiseVelocities()} & Generates a random value for \texttt{target\_velocity\_y} and \texttt{target\_velocity\_x}, with the value between \texttt{MIN\_VELOCITY\_Y}, \texttt{MAX\_VELOCITY\_Y}, and \texttt{MIN\_VELOCITY\_X}, \texttt{MAX\_VELOCITY\_X}.\\
        \hline
        \texttt{move()} & Updates the particle's \texttt{x} and \texttt{y} positions based on its \texttt{velocity\_x} and \texttt{velocity\_y} and the simulation time since the last frame. Also updates the particle's velocities, based on the difference between the current velocity and the target velocity, with the \texttt{VELOCITY\_CHANGE\_RATE} multiplier.\\
        \hline
        \texttt{draw()} & Draws the particle based on its \texttt{x} and \texttt{y} positions, and \texttt{radius}, with the colour from the \texttt{PARTICLE\_COLOUR} constant, using the Pygame library's \texttt{pygame.draw.circle()} function.\\
        \hline
    \end{tabularx}
    \caption{Methods and their descriptions from the Bubble class of the Bubble Backscatter Generator program.}
    \label{table:simbubbleclassfuncs}
\end{table}

\subsection{Python Code}
\label{sim_code}
\lstinputlisting[language=Python, caption={The Python code for the Bubble Backscatter Simulator, commit version: \texttt{34b9a82} from \cite{sidharthshanmugamBubblebackscattersimulator2024}.}]{code/00_bubble-backscatter-simulator/app.py}
